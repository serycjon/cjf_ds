%%% Template originaly created by Karol Kozioł (mail@karol-koziol.net) and modified for ShareLaTeX use

\documentclass[a4paper,11pt]{article}

\usepackage[T1]{fontenc}
\usepackage[utf8]{inputenc}
\usepackage{graphicx}
\usepackage{xcolor}

\renewcommand\familydefault{\sfdefault}
\usepackage{tgheros}
\usepackage[defaultmono]{droidmono}

\usepackage{amsmath,amssymb,amsthm,textcomp}
\usepackage{enumerate}
\usepackage{multicol}
\usepackage{tikz}

\usepackage{geometry}
\geometry{total={210mm,297mm},
left=25mm,right=25mm,%
bindingoffset=0mm, top=20mm,bottom=20mm}


\linespread{1.3}

\newcommand{\linia}{\rule{\linewidth}{0.5pt}}

% custom theorems if needed
\newtheoremstyle{mytheor}
    {1ex}{1ex}{\normalfont}{0pt}{\scshape}{.}{1ex}
    {{\thmname{#1 }}{\thmnumber{#2}}{\thmnote{ (#3)}}}

\theoremstyle{mytheor}
\newtheorem{defi}{Definition}

% my own titles
\makeatletter
\renewcommand{\maketitle}{
\begin{center}
\vspace{2ex}
{\huge \textsc{\@title}}
\vspace{1ex}
\\
\linia\\
\@author \hfill \@date
\vspace{4ex}
\end{center}
}
\makeatother
%%%

% custom footers and headers
\usepackage{fancyhdr}
\pagestyle{fancy}
\lhead{}
\chead{}
\rhead{}
\lfoot{DS CP 2}
\cfoot{}
\rfoot{Strana \thepage}
\renewcommand{\headrulewidth}{0pt}
\renewcommand{\footrulewidth}{0pt}
%

% code listing settings
\usepackage{listings}
\lstset{
    language=SQL,
    basicstyle=\ttfamily\small,
    aboveskip={1.0\baselineskip},
    belowskip={1.0\baselineskip},
    columns=fixed,
    extendedchars=true,
    breaklines=true,
    tabsize=4,
    prebreak=\raisebox{0ex}[0ex][0ex]{\ensuremath{\hookleftarrow}},
    frame=lines,
    showtabs=false,
    showspaces=false,
    showstringspaces=false,
    keywordstyle=\color[rgb]{0.627,0.126,0.941},
    commentstyle=\color[rgb]{0.133,0.545,0.133},
    stringstyle=\color[rgb]{01,0,0},
    numbers=left,
    numberstyle=\small,
    stepnumber=1,
    numbersep=10pt,
    captionpos=t,
    escapeinside={\%*}{*)}
}
\renewcommand{\lstlistingname}{Kód}

%%%----------%%%----------%%%----------%%%----------%%%

\begin{document}

\title{DS - Česká Jezdecká Federace}

\author{Ondráčková, Šerých}

\date{20/05/2014}

\maketitle

\section*{Triggery}

V našem databázovém systému potřebujeme kontrolovat, jestli je v každém registrovaném týmu právě tolik koní a tolik lidí, kolik je v dané soutěžní kategorii povoleno. To se ukázalo jako netriviální omezení, které je potřeba kontrolovat triggerem. Rozhodli jsme se, že se počty zkontrolují ve chvíli, kdy se má přiřadit tým k závodu. Tím se uzavře registrace týmu a již dále nelze měnit jeho obsazení. Samotný trigger je ve výpisu~\ref{list:first}\ldots{}

\begin{lstlisting}[label={list:first},caption=trigger check pocty.]
CREATE OR REPLACE FUNCTION check_pocty() RETURNS TRIGGER AS '
	DECLARE
		pocet_koni_povoleny INTEGER;
		pocet_lidi_povoleny INTEGER;
		pocet_koni_realny INTEGER;
		pocet_jezdcu INTEGER;
		pocet_lidi_realny INTEGER;
		check_tym_id INTEGER;
	BEGIN
	
	IF (TG_OP = ''INSERT'') AND NEW.zavod_id IS NOT NULL THEN
		RAISE EXCEPTION 
			''Tym lze priradit k zavodu az po jeho naplneni!'';
	END IF;
	IF NEW.zavod_id IS NULL THEN
		RETURN NEW;
	END IF;
	check_tym_id := NEW.tym_id;
	SELECT pocet_koni, pocet_prisedicich
		INTO pocet_koni_povoleny, pocet_lidi_povoleny
		FROM kategorie
		INNER JOIN tymy ON kategorie.kategorie_id = tymy.kategorie_id
		WHERE tymy.tym_id = check_tym_id;
	SELECT COUNT(kone_kun_id) INTO pocet_koni_realny
		FROM tymy_has_kone GROUP BY tymy_tym_id
		HAVING tymy_tym_id = check_tym_id;

	SELECT COUNT(osoby_osoba_id) INTO pocet_lidi_realny
		FROM tymy_has_osoby WHERE NOT je_jezdec GROUP BY tymy_tym_id
		HAVING tymy_tym_id = check_tym_id;
	SELECT COUNT(osoby_osoba_id) INTO pocet_jezdcu
		FROM tymy_has_osoby WHERE je_jezdec GROUP BY tymy_tym_id 
		HAVING tymy_tym_id = check_tym_id; 

	IF NOT pocet_koni_povoleny=pocet_koni_realny
		OR pocet_koni_realny IS NULL THEN
			RAISE EXCEPTION ''Spatny pocet koni!'';
	ELSEIF NOT pocet_lidi_povoleny=pocet_lidi_realny
		OR pocet_lidi_realny IS NULL THEN
			RAISE EXCEPTION ''Spatny pocet prisedicich!'';
	ELSEIF NOT pocet_jezdcu = 1 OR pocet_jezdcu IS NULL THEN
		RAISE EXCEPTION ''Spatny pocet jezdcu!'';
	END IF;

	RETURN NEW;
	END
	'
	LANGUAGE plpgsql;

CREATE TRIGGER trig_pocty
   BEFORE UPDATE OR INSERT ON tymy
   FOR EACH ROW
   EXECUTE PROCEDURE check_pocty();
\end{lstlisting}

\section*{Uložené procedury}
V naší databázi používáme uloženou proceduru prirad\_startovni\_cisla(zavod\_id) (ve výpisu~\ref{list:second}) která přiřadí všem týmům zaregistrovaným v daném závodu náhodná startovní čísla
od jedné do počtu týmů. Nejprve jsme k tomuto účelu chtěli použít SEQUENCE, ale ukázalo se, že to není praktické řešení. Naše procedura tedy nejdřív SELECTuje
všechny týmy a náhodně je seřadí a tyto výsledky pak prochází po řádkách a do startovního čísla ukládá postupně inkrementovanou lokální proměnnou.


\begin{lstlisting}[label={list:second},caption=Uložená procedura prirad\_startovni\_cisla\(\)]
CREATE OR REPLACE FUNCTION prirad_startovni_cisla(zavod INTEGER) RETURNS boolean AS '
	DECLARE
		i RECORD;
		update_count INTEGER;
		cislo INTEGER;
	BEGIN
		cislo := 1;
		FOR i IN SELECT tym_id FROM tymy WHERE zavod_id = zavod ORDER BY random() LOOP
			RAISE DEBUG ''updating id %'', i.tym_id;
			UPDATE tymy SET startovni_cislo = cislo WHERE tym_id = i.tym_id;
			GET DIAGNOSTICS update_count = ROW_COUNT;
			cislo := cislo+1;
			IF update_count < 1 THEN
				RETURN FALSE;
			END IF;
		END LOOP;
		RETURN TRUE;
	END
	'
	LANGUAGE plpgsql;
\end{lstlisting}

\section*{Statistiky}
Česká Jezdecká Federace každoročně vyhlašuje jezdce,
který strávil na kolbišti nejvíce času. Pro tuto statistiku jsme vytvořili následující dotaz (ve výpisu~\ref{list:third})
\begin{lstlisting}[label={list:second},caption=Uložená procedura prirad\_startovni\_cisla\(\)]
CREATE OR REPLACE FUNCTION prirad_startovni_cisla(zavod INTEGER) RETURNS boolean AS '
	DECLARE
		i RECORD;
		update_count INTEGER;
		cislo INTEGER;
	BEGIN
		cislo := 1;
		FOR i IN SELECT tym_id FROM tymy WHERE zavod_id = zavod ORDER BY random() LOOP
			RAISE DEBUG ''updating id %'', i.tym_id;
			UPDATE tymy SET startovni_cislo = cislo WHERE tym_id = i.tym_id;
			GET DIAGNOSTICS update_count = ROW_COUNT;
			cislo := cislo+1;
			IF update_count < 1 THEN
				RETURN FALSE;
			END IF;
		END LOOP;
		RETURN TRUE;
	END
	'
	LANGUAGE plpgsql;
\end{lstlisting}

\end{document}
